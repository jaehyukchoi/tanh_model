\documentclass[authoryear,reqno,12pt,a4paper]{elsarticle}  
%\documentclass[a4paper,12pt,reqno]{amsart}
%\pdfoutput=1

\usepackage{amsmath,natbib,graphicx,hyperref,soul}
\usepackage[left=1in,right=1in,top=1.4in,bottom=1.2in]{geometry}
\renewcommand{\baselinestretch}{1.5}

\hypersetup{
	pdftitle={Black-Scholes and Normal Hybrid Stochastic Volatility Model},
	pdfauthor={Jaehyuk Choi and Gong Li},%pdfsubject={Subject??},
	pdfkeywords={Black-Scholes, Bachelier, Normal model},
	colorlinks=true,
	linkcolor=red,
	citecolor=blue,
	urlcolor=blue,
	bookmarksnumbered=true,%Show chapter/section number in PDF thumbnails
	pdfstartview=%Acrobat reader's default zoom setting (vs Fit)
}

\newcommand{\rhoc}{\rho_\ast}
\newcommand{\vov}{\alpha}
\newcommand{\norm}{\textsc{n}}
\newcommand{\bs}{\textsc{bs}}
\newcommand{\cev}{\textsc{cev}}
\newcommand{\ncx}{\textsc{ncx2}}
\newcommand{\AV}{\textsc{av}}

\newcommand{\qtext}[2][\quad]{#1\text{#2}#1}
\newcommand{\todo}[1]{\hl{[\,#1\,]}}
\newcommand{\citewiki}[1]{(\href{https://en.wikipedia.org/wiki/#1}{\textmd{\textsc{WikipediA}}})}

\journal{PHBS Working Paper}  %European Journal of Operational Research
\bibliographystyle{elsarticle-harv}%\biboptions{authoryear}

\begin{document}
\begin{frontmatter}

\title{Black-Scholes and Normal Hybrid Stochastic Volatility Model}

\author[phbs]{Jaehyuk Choi\corref{corrauthor}}
\ead{jaehyuk@phbs.pku.edu.cn}

\author[phbs]{Gong Li}
\ead{1801212845@pku.edu.cn}

\address[phbs]{Peking University HSBC Business School, Shenzhen, China}

\cortext[corrauthor]{Corresponding author \textit{Tel:} +86-755-2603-0568, \textit{Address:} Rm 755, Peking University HSBC Business School, University Town, Nanshan, Shenzhen 518055, China}

\date{Nov 26, 2019}

\begin{abstract}
We propose an option pricing model where the dynamics of asset price gradually changes from normal model to Black-Scholes model as the asset price increases.
\end{abstract}

\begin{keyword}
	stochastic volatility, Black-Scholes model, Bachelier model, SABR model
\end{keyword}
\end{frontmatter}

%\linenumbers

\section{Introduction}
The dynamics of asset price under the Black-Scholes model is given by the geometric Brownian Motion,
$$ dF_t = \sigma F_t\, dW_t. $$  
and that under the Bachelier (normal) model is given by
$$ dF_t = \sigma_\norm\, dW_t. $$  

Interest rate is known to follow the BSM model when the rate $F_t$ is well below $h$ and the normal model when $F_t$ is well above $h$ for some reference rate $h$. Therefore, we want to construct a hybrid model.

Assume
$$ dF_t = \sigma C(F_t) dW_t. $$  
The local volatility function $C(x)$ should satisfy the following properties:
\begin{itemize}
	\item $C(x)\approx x$ when $x$ is small. Therefore, $\sigma$ plays the role of the BSM volatility when $x$ is small.
	\item $C(x)\rightarrow h$ when $x$ is large. The parameter $h$ is understood as a scale of $F_t$. Therefore, $\sigma_N = h \sigma$ plays the role of the normal volatility when $x$ is large.
\end{itemize}

\section{Previously known methods}
When $C(x)=x^\beta$, the model is known as the constant elasticity of variance (CEV) model:
$$ \frac{d F_t}{F_t^\beta} = \sigma dW_t. $$
It is well known that the option price is given as
$$ C_\beta(K,F_0,\sigma) = F_0 \, Q_\ncx\left(\frac{q^2(K)}{\sigma^2 T}; \,2+\frac1{1-\beta},\frac{q^2(F_0)}{\sigma^2 T}\right) - K \, P_\ncx\left(\frac{q^2(F_0)}{\sigma^2 T};\, \frac1{1-\beta},\frac{q^2(K)}{\sigma^2 T}\right),
$$
where $q(x) = x^{1-\beta}/(1-\beta)$, and $P_\ncx(x;k, x_0)$ and $Q_\ncx(x;k,x_0)$ are the cumulative probability density (CDF) and survival functions, respectively, of the noncentral chi-squared distribution with degree of freedom $k$ and noncentrality parameter $x_0$.

\citet{hagan1999equiv} derived the approximate equivalent BSM volatility for general $C(x)$. The solution is somewhat accurate near-the-money options.


\section{Ideas}
\subsection{Candidate 1}
For our first candidate, we consider 
$$C(x) = \frac{hx}{h+x}.$$ 
This function satisfies the requirements.
Ito's lemma leads to
\begin{gather*}
\left(\frac1{F_t}+\frac{1}{h}\right) dF_t = \sigma dW_t \\
d (\log F_t + F_t/h) = \sigma dW_t - \frac{\sigma^2 h^2}{2(F_t+h)^2}dt
\end{gather*}
The integral in the left hand side is related to the Lambert W function $W(z)$  \citewiki{Lambert_W_function} defined by 
$$ z = W(z\,e^z) \qtext{or}  W^{-1}(z) = z\,e^{z}. $$
When $z$ is small, $W(z)\approx z$ and, when $z$ is large, $W(z)=\log z$.

To simplify the calculation, we assume that $F_t$ is not far way from the initial point $F_0$. 
$$ d (\log F_t + F_t/h) = \sigma dW_t - \frac{\sigma^2 h^2}{2(F_0+h)^2}dt
$$
Then,
$$ F_T e^{F_T/h} = F_0 e^{F_0/h} \exp\left(\sigma W_T - \frac{\sigma^2 h^2T}{2(F_0+h)^2} \right)
$$

$$ F_T = h W\left( (F_0/h) e^{F_0/h} \exp\left( \sigma W_T - \frac{\sigma^2 h^2 T}{2(F_0+h)^2} \right)\right)
$$
When $F_0 \ll h$, the dynamics follows the BSM model ($W(z)\approx z$):
$$ F_T = F_0 \exp\left( \sigma W_T - \frac{\sigma^2 T}{2} \right).
$$
When $F_0 \gg h$, the dynamics follows the normal model ($W(z)\approx \log z$):
$$ F_T = F_0 + \sigma W_T.
$$

Can we price the option quickly?
$$C(K) = E\left( (F_T-K)^+\right)
$$



\subsection{Candidate 2}
We consider 
$$ C(x) = h\tanh(x/h)$$
which also satisfies the requirements.

Using
$$ \int \frac{dx}{h\tanh(x/h)} = \int \frac{\cosh(x/h)}{h\sinh(x/h)} dx = \log(\sinh(x/h)) + \text{const.},
$$
we obtain:
\begin{gather*}
\frac{\cosh(F_t/h)}{h\sinh(F_t/h)} dF_t = \sigma dW_t \\
d \log(\sinh(F_t/h)) = \sigma dW_t + \frac{\sigma^2}{2} (\tanh^2(F_t/h)-1) dt.
\end{gather*}
\todo{Check the computation.}

Again, assuming that $F_t\approx F_0$,
\begin{gather*}
\sinh(F_T/h) = \sinh(F_0/h) \exp \left(
\sigma W_T + \frac{\sigma^2 T}{2} (\tanh^2(F_0/h)-1)
\right) \\
F_T = h \sinh^{-1} \left( \sinh(F_0/h) \exp \left(\sigma W_T + \frac{\sigma^2 T}{2} (\tanh^2(F_0/h)-1)
\right) \right) \\
\end{gather*}

We find the similar asymptotic behavior in this candidate too. Reminded that $\sinh^{-1} z = \log(z+\sqrt{1+z^2})$. Therefore, $\sinh z \approx z$ when $z\ll 1$, $\sinh z \approx \log(2z)$ when $z\gg 1$.
When $F_0 \ll h$, the dynamics follows the BSM model:
$$ F_T = F_0 \exp\left( \sigma W_T - \frac{\sigma^2 T}{2} \right).
$$
When $F_0 \gg h$, the dynamics follows the normal model:
$$ F_T = F_0 + \sigma W_T.
$$

\section{Research questions}
\begin{itemize}
	\item How do we price the option? Any approximation?
	\item How can we run Monte-Carlo simulation efficiently? (It will be the subject of the \textit{Applied Stochastic Processes} course.)
\end{itemize}

\section{Extension}
Eventually, I want to solve the stochastic volatility model with the local volatility function $C(x)$:
$$ dF_t = \sigma_t C(F_t) dW_t,\quad \sigma_t = \vov \sigma_t dZ_t, \qtext{and} dW_t dZ_t = \rho\, dt. $$ 
When $C(x)=x^\beta$, the model is known as stochastic-alpha-beta-rho (SABR) model~\citep{hagan2002sabr}.

\bibliography{SV_Z}
%\bibliography{../../@Bib/SV_Z}

\end{document}